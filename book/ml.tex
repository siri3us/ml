\documentclass{book}
\usepackage[english, russian]{babel}
\usepackage[utf8]{inputenc}
\usepackage[margin=0.75in]{geometry}
\usepackage{paralist}
\usepackage{amsthm, amsmath, amsfonts, amssymb}
%\usepackage{bbold}
\usepackage{dsfont}
\usepackage{mathtools}
\usepackage{hyperref}
\usepackage{graphicx}
\usepackage{listings}
\usepackage{algorithm}
\usepackage{algpseudocode} 
\usepackage{multirow}
\usepackage{comment}

% Next goes different graphic related issues are discussed
\usepackage{xcolor, colortbl}
\usepackage{tikz}
\usepackage{xifthen}
\usetikzlibrary{arrows}
\usetikzlibrary{positioning}
\usepackage{pgf}
\usepackage{pgfplots}

\definecolor{grey1}{RGB}{192,192,192}
\definecolor{lemonchiffon}{RGB}{255,250,205}
\definecolor{yellowgreen}{RGB}{154,205,50}
\definecolor{chocolate}{RGB}{210,105,30}
\definecolor{purple}{RGB}{128,0,128}

\usepackage{sectsty}
%\subsectionfont{\color{blue}}
\subsubsectionfont{\color{red}}  % sets colour of sections

\usepackage{caption,subcaption}

\usepackage{bm}

\usepackage{indentfirst} % Отступ в начале chapter, section, subsection и т.д.

\newtheorem{theorem}{Теорема}
\numberwithin{theorem}{chapter}

\newtheorem{statement}{Утверждение}
\numberwithin{statement}{chapter}

\newtheorem{lemma}{Лемма}
\numberwithin{lemma}{chapter}

\newtheorem{consequence}{Следствие}

\theoremstyle{definition}
\newtheorem{task}{Задание}
\numberwithin{task}{chapter}

\theoremstyle{remark}
\newtheorem{example}{Пример}
\numberwithin{example}{chapter}

\newtheorem{assumption}{Предположение}

\theoremstyle{definition}
\newtheorem{definition}{Определение}
\numberwithin{definition}{chapter}

\theoremstyle{remark}
\newtheorem{note}{Замечание}
\theoremstyle{remark}
\newtheorem{lyrics}{Лирическое отступление}
\numberwithin{lyrics}{section}

%\renewenvironment{itemize}[1]{\begin{compactitem}#1}{\end{compactitem}}
%\renewenvironment{enumerate}[1]{\begin{compactenum}#1}{\end{compactenum}}
%\renewenvironment{description}[0]{\begin{compactdesc}}{\end{compactdesc}}

\newcommand{\param}[1]{\textbf{#1}}
\newcommand{\numberof}[1]{\#[\text{#1}]}
\newcommand{\underquestion}[1]{\textbf{#1 (Check!)}}
\newcommand{\TODO}[1]{\textbf{[TODO:#1]}}

% Комманды для комментариев
\newcommand{\ignore}[1]{}
\newcommand{\hidden}[1]{}
\newcommand{\translation}[1]{}

\DeclareMathAlphabet{\mathbbold}{U}{bbold}{m}{n}


\begin{document}
\title{Глубинное Обучение}
\author{Иванов Александр \\ ИППИ РАН}
\date{}
\maketitle
	
\tableofcontents
	
% The list of general commands
\newcommand{\PI}{3.141592654}
\newcommand{\Sum}{\sum\limits}
\newcommand{\Int}{\int\limits}
\newcommand{\Prod}{\prod\limits}
\newcommand{\Max}{\max\limits}
\newcommand{\Min}{\min\limits}
\newcommand{\Var}{\mathbb{V}}
\newcommand{\Exp}[1]{\mathbb{E}[#1]}
\newcommand{\argmax}{\arg\max}
\newcommand{\Cov}{\text{Cov}}
\newcommand{\makebold}[1]{\boldsymbol{#1}}

% The list of general mathematical commands

% The list of coding.text commands
\newcommand{\Priv}{U}
\newcommand{\Morder}{P}
\newcommand{\One}{\mathds{1}}
\newcommand{\Zero}{\mathds{0}}
\newcommand{\Field}{F}
\newcommand{\Code}{\mathbb{C}}
\newcommand{\Id}{\mathbbm{1}}

% The list of PHY.tex commands
\newcommand{\received}[1]{\hat{#1}}
\newcommand{\proba}[1]{\mathbb{P}_{#1}}
\newcommand{\real}[1]{\mathbb{R}^{#1}}
\newcommand{\dmin}{d_{\min}}
\newcommand{\decisionf}[1]{g(#1)}

\newcommand{\energy}{\mathcal{E}}
\newcommand{\power}{P}

\newcommand{\decfunc}{g}
\newcommand{\decreg}{D}
\renewcommand{\vec}{\bm}
\newcommand{\px}[1]{p_{\vec{x}}(#1)}
\newcommand{\ps}[1]{p_{\vec{s}}(#1)}



%\newcommand{\rate}{R}
\newcommand{\brate}{R_b}
\newcommand{\srate}{R_s}
\newcommand{\Tsym}{T_s}
\newcommand{\Tbit}{T_b}
\newcommand{\N}{N}
\newcommand{\M}{M}
\newcommand{\sign}{\text{sign}}


\def\MYdef{\mathrel{\stackrel{\rm def}=}}

\newcommand{\Real}[1]{\Re\left\{#1\right\}}
\newcommand{\Imag}[1]{\Im\left\{#1\right\}}

\newcommand{\Energy}{\mathcal{E}}
\newcommand{\Const}{\mathfrak{A}}

% Различные преобразования
\newcommand{\Freq}{f}
\newcommand{\Complex}[1]{\tilde{#1}}
\newcommand{\Hilbert}[1]{\hat{#1}}
\newcommand{\fourier}[1]{\mathcal{F}[#1](\Freq)}
\newcommand{\bfourier}[1]{\mathcal{F}^{-1}[#1](t)}
\newcommand{\Lower}[2]{\mathcal{L}_{#2}[#1](t)}

\newcommand{\fupp}{f_{u}}
\newcommand{\fdis}{f_{d}}

\newcommand{\DeltaFreq}{\Delta \Freq}
\newcommand{\DeltaTime}{\Delta t}

\newcommand{\NumberOfSubcarriers}{N_c}
\newcommand{\Nc}{\NumberOfSubcarriers}
\newcommand{\Nfft}{N}
\newcommand{\FreqIndex}{n}
\newcommand{\FI}{\FreqIndex}
\newcommand{\TimeIndex}{k}
\newcommand{\TI}{\TimeIndex}

\newcommand{\OfdmSubcarrier}[1]{\ifthenelse{\isempty{#1}}{\Freq_\FreqIndex}{\Freq_{#1}}}
\newcommand{\OfdmSubcarriers}{\{\Freq_\FreqIndex\}_{\FreqIndex=0}^{\NumberOfSubcarriers-1}}

\newcommand{\TxSignal}{x}
\newcommand{\OfdmSignal}{\TxSignal}
\newcommand{\RealOfdmSignal}{\OfdmSignal}
\newcommand{\CompOfdmSignal}{\tilde{\TxSignal}}

\newcommand{\OfdmSubsignal}[1]{\ifthenelse{\isempty{#1}}{\OfdmSignal_{\FreqIndex}(t)}{\OfdmSignal_{#1}(t)}}
\newcommand{\RealOfdmSubsignal}[1]{\OfdmSubsignal{#1}}
\newcommand{\CompOfdmSubsignal}[1]{\ifthenelse{\isempty{#1}}{\CompOfdmSignal_{\FreqIndex}(t)}{\CompOfdmSignal_{#1}(t)}}

\newcommand{\OfdmSubsignals}{\{\OfdmSignal_{\FreqIndex}(t)\}_{\FreqIndex=0}^{\NumberOfSubcarriers-1}}
\newcommand{\RealOfdmSubsignals}{\OfdmSubsignals}
\newcommand{\CompOfdmSubsignals}{\{\CompOfdmSignal_{\FreqIndex}(t)\}_{\FreqIndex=0}^{\NumberOfSubcarriers-1}}

\newcommand{\pulse}{g}
\newcommand{\symb}{A}


% The list of PROBLEMS.tex commands

% LTE
\newcommand{\BSs}{\mathbb{BS}}
\newcommand{\UEs}{\mathbb{MS}}
\newcommand{\BWs}{\mathbb{BW}}
\newcommand{\NoBS}{\mathcal{B}}
\newcommand{\NoUE}{\mathcal{U}}
\newcommand{\NoBW}{\mathcal{W}}

\newcommand{\Tres}{T_{res}}
\newcommand{\Tin}{T_{in}}
\newcommand{\tres}{t_{res}}
\newcommand{\tin}{t_{in}}
\newcommand{\age}{h}
\newcommand{\recv}[1]{S_{#1}}
\newcommand{\PER}{p}

\newcommand{\BlockSize}{B}
\newcommand{\ActiveBlockSize}{N}
\newcommand{\QueueSize}{Q}

\newcommand{\packet}{\boldsymbol{c}}
\newcommand{\nativepackets}{\boldsymbol{M}}
\newcommand{\allpackets}{\boldsymbol{X}}
\newcommand{\rate}{r}
\newcommand{\group}[1]{\mathcal{G}_{#1}}

\newcommand{\weight}{w}
\newcommand{\prob}{\mathbb{P}}

%Simulink
\newcommand{\fcn}[1]{\texttt{#1}}
\newcommand{\variable}[1]{\texttt{#1}}
\newcommand{\question}[1]{\textbf{#1}}
\newcommand{\command}[1]{\texttt{#1}}

%\makeatletter
%\def\eprob{\@ifnextchar[{\@condeprob}{\@notcondeprob}}
%\def\@notcondeprob{\mathbb{P}_{e}}
%\def\@condeprob[#1]{\mathbb{P}_{e|#1}}
%\makeatother

%\makeatletter
%\def\prob{\@ifnextchar[{\@condprob}{\@notcondprob}}
%\def\@condprob[#1]#2{\mathbb{P}\{#2|#1\}}
%\def\@notcondprob#1{\mathbb{P}\{#1\}}
%\makeatother


% Neural network
% Convolutional Networks

\newcommand{\partder}[2]{\frac{\partial #1 }{\partial #2}}

\newcommand{\Loss}{\mathcal{L}}

\newcommand{\Input}{I}
\newcommand{\Output}{O}

\newcommand{\Weights}{W}
\newcommand{\Filter}{F}

\newcommand{\FilterAnchor}{\mathbb{d}}
\newcommand{\AffectField}{A}
\newcommand{\ReceptiveField}{R}
\newcommand{\FilterSize}{\boldsymbol{f}}

\newcommand{\StrideSize}{\boldsymbol{s}}

\newcommand{\NumOfSamples}{\mathcal{N}}
\newcommand{\NumOfChannel}{\mathcal{C}}
\newcommand{\NumOfFilters}{\mathcal{F}}



\chapter{Полносвязные Сети}

\chapter{Сверточные Сети}

\section{Обратное распространение ошибки}

\subsection{Случай $\text{stride} = 1$}

\begin{equation}
\partder{\Loss}{\Input[c, h, w]}= \Sum_{(i, j) \in \FilterCoverageSet(h, w)} 
\partder{\Output[c, i, j]}{\Input[c, h, w]} \cdot \partder{\Loss}{\Output[c, i, j]}
\end{equation}

\begin{align}
\FilterCoverageSet(h)& = \{i \colon \max\{0, h - \FilterSize[h]\} \le i \le \min\{h, H - \FilterSize[h]\}\}, \\
\FilterCoverageSet(w)& = \{j \colon \min\{0, w - \FilterSize[w]\} \le j \le \max\{w, W - \FilterSize[w]\}\}.
\end{align}

\begin{equation}
\FilterCoverageSet(h, w) \triangleq \{(i, j)\colon i \in \FilterCoverageSet(h), j\in \FilterCoverageSet(w)\}.
\end{equation}


\begin{equation}
\label{eq:conv:backprop:1}
\partder{\Loss}{\Input[c, h, w]}= \Sum_{(i, j) \in \FilterCoverageSet(h, w)} 
\partder{\Output[c, i, j]}{\Input[c, h, w]} \cdot \partder{\Loss}{\Output[c, i, j]} = 
\Sum_{(i, j) \in \FilterCoverageSet(h, w)} W[c, h - i, w - j] \partder{\Loss}{\Output[c, i, j]}
\end{equation}
Уже здесь можно заметить, что нахождение производных по $\Input$ напоминает свертку производных по выходу $\partder{\Loss}{\Output}$
с транспонированной (вдоль второго и третьего измерений) матрицей весов $W$.

Рассмотрим пока для определенности только точки $(h, w)$ из диапазона $(h, w) \in [\FilterSize[h] - 1, H - \FilterSize[h]] \times [\FilterSize[w] - 1, W - \FilterSize[w]]$. В этом случае сумму из~\eqref{eq:conv:backprop:1} можно записать в виде
\begin{equation}
\label{eq:conv:backprop:2}
\partder{\Loss}{\Input[c, h, w]} = \Sum_{\substack{i \in [h - \FilterSize[h] + 1, h] \\ j \in [w - \FilterSize[w] + 1, w]}} W[c, h - i, w - j] \partder{\Loss}{\Output[c, i, j]},
\end{equation}
не содержащим граничных условий на $i$ и $j$. Далее введем транспонированную вдоль второго и третьего измерений матрицу весов $W'$:
\begin{equation}
W'[i, j] = W[c, \FilterSize[h] - 1 - i, \FilterSize[w] - 1 - j], \quad c \in [0, \NumOfFilters - 1], i \in [0, \FilterSize[h] - 1], j \in [0, \FilterSize[w] - 1].
\end{equation}
В таком случае~\eqref{eq:conv:backprop:2} можно далее записать в виде
\begin{equation}
\partder{\Loss}{\Input[c, h, w]} = \Sum_{\substack{i \in [h - \FilterSize[h] + 1, h] \\ j \in [w - \FilterSize[w] + 1, w]}} W'[c, i, j] \partder{\Loss}{\Output[c, i, j]}.
\end{equation}

Можно в принципе избавиться от граничных условий и записать для любых $(h, w) \in [0, H - 1] \times 0, W - 1]$, если положить, что значения по индексам, отсутствующим в матрице $\partial \Loss / \partial I$, равны нулю:
\begin{equation}
\partder{\Loss}{\Input[c, h, w]} = \Sum_{\substack{i \in [h - \FilterSize[h] + 1, h] \\ j \in [w - \FilterSize[w] + 1, w]}} W'[c, i, j] \partder{\Loss}{\Output[c, i, j]}.
\end{equation}
Фактически это эквивалентно тому, что матрица $\partder{\Loss}{\Output[c, i, j]}$ дополняется $\FilterSize[h] - 1$ вдоль второго измерения и $\FilterSize[w] - 1$ вдоль третьего измерения с каждой стороны. Полученная матрица имеет размер $(H + \FilterSize[h] - 1) \times (W + \FilterSize[w] - 1)$. 

\subsection{Случай $\text{stride} = 2$}

\begin{equation}
\partder{\Loss}{\Input[c, h, w]}= \Sum_{(i, j) \in \FilterCoverageSet(h, w)} 
\partder{\Output[c, i, j]}{\Input[c, h, w]} \cdot \partder{\Loss}{\Output[c, i, j]}
\end{equation}


\bibliographystyle{ugost2008}
\bibliography{../bib/phy.bib}

\end{document}