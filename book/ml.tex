\documentclass{report}
\usepackage[english, russian]{babel}
\usepackage[utf8]{inputenc}
\usepackage[margin=0.75in]{geometry}
\usepackage{paralist}
\usepackage{amsthm, amsmath, amsfonts, amssymb}
%\usepackage{bbold}
\usepackage{dsfont}
\usepackage{xspace}
\usepackage{mathtools}
\usepackage{hyperref}
\usepackage{graphicx}
\usepackage{listings}
\usepackage{algorithm}
\usepackage{algpseudocode} 
\usepackage{multirow}
\usepackage{comment}

% Next goes different graphic related issues are discussed
\usepackage{xcolor, colortbl}
\usepackage{tikz}
\usepackage{xifthen}
\usetikzlibrary{arrows}
\usetikzlibrary{positioning}
\usepackage{pgf}
\usepackage{pgfplots}


% Python style for highlighting
\usepackage{listings}
\definecolor{deepblue}{rgb}{0,0,0.5}
\definecolor{deepred}{rgb}{0.6,0,0}
\definecolor{deepgreen}{rgb}{0,0.5,0}
\DeclareFixedFont{\ttb}{T1}{txtt}{bx}{n}{8} % for bold
\DeclareFixedFont{\ttm}{T1}{txtt}{m}{n}{8}  % for normal
\newcommand\pythonstyle{\lstset{
		language=Python,
		basicstyle=\ttm,
		otherkeywords={self},             % Add keywords here
		keywordstyle=\ttb\color{deepblue},
		emph={MyClass,__init__},          % Custom highlighting
		emphstyle=\ttb\color{deepred},    % Custom highlighting style
		stringstyle=\color{deepgreen},
		frame=tb,                         % Any extra options here
		showstringspaces=false            % 
	}}

	
% Python environment
\lstnewenvironment{python}[1][]
{
	\pythonstyle
	\lstset{#1}
}
{}
	

\definecolor{grey1}{RGB}{192,192,192}
\definecolor{lemonchiffon}{RGB}{255,250,205}
\definecolor{yellowgreen}{RGB}{154,205,50}
\definecolor{chocolate}{RGB}{210,105,30}
\definecolor{purple}{RGB}{128,0,128}

\usepackage{sectsty}
%\subsectionfont{\color{blue}}
\subsubsectionfont{\color{red}}  % sets colour of sections

%\usepackage{caption,subcaption}

\usepackage{bm}

\usepackage{indentfirst} % Отступ в начале chapter, section, subsection и т.д.

\newtheorem{theorem}{Теорема}
\numberwithin{theorem}{chapter}

\newtheorem{statement}{Утверждение}
\numberwithin{statement}{chapter}

\newtheorem{lemma}{Лемма}
\numberwithin{lemma}{chapter}

\newtheorem{consequence}{Следствие}

\theoremstyle{definition}
\newtheorem{task}{Задание}
\numberwithin{task}{chapter}

\theoremstyle{remark}
\newtheorem{example}{Пример}
\numberwithin{example}{chapter}

\newtheorem{assumption}{Предположение}

\theoremstyle{definition}
\newtheorem{definition}{Определение}
\numberwithin{definition}{chapter}

\theoremstyle{remark}
\newtheorem{note}{Замечание}
\theoremstyle{remark}
\newtheorem{lyrics}{Лирическое отступление}
\numberwithin{lyrics}{section}

%\renewenvironment{itemize}[1]{\begin{compactitem}#1}{\end{compactitem}}
%\renewenvironment{enumerate}[1]{\begin{compactenum}#1}{\end{compactenum}}
%\renewenvironment{description}[0]{\begin{compactdesc}}{\end{compactdesc}}

\newcommand{\param}[1]{\textbf{#1}}
\newcommand{\numberof}[1]{\#[\text{#1}]}
\newcommand{\underquestion}[1]{\textbf{#1 (Check!)}}
\newcommand{\TODO}[1]{\textbf{[TODO:#1]}}

% Комманды для комментариев
\newcommand{\ignore}[1]{}
\newcommand{\hidden}[1]{}
\newcommand{\translation}[1]{}

\DeclareMathAlphabet{\mathbbold}{U}{bbold}{m}{n}


\begin{document}
\title{Глубинное Обучение}
\author{Иванов Александр \\ ИППИ РАН}
\date{}
\maketitle
\tableofcontents
	
% The list of general commands
\newcommand{\PI}{3.141592654}
\newcommand{\Sum}{\sum\limits}
\newcommand{\Int}{\int\limits}
\newcommand{\Prod}{\prod\limits}
\newcommand{\Max}{\max\limits}
\newcommand{\Min}{\min\limits}
\newcommand{\Var}{\mathbb{V}}
\newcommand{\Exp}{\mathbb{E}}
\newcommand{\argmax}{\arg\max}
\newcommand{\Cov}{\text{Cov}}
\newcommand{\makebold}[1]{\boldsymbol{#1}}

% The list of general mathematical commands

% The list of coding.text commands
\newcommand{\Priv}{U}
\newcommand{\Morder}{P}
\newcommand{\One}{\mathds{1}}
\newcommand{\Zero}{\mathds{0}}
\newcommand{\Field}{F}
\newcommand{\Code}{\mathbb{C}}
\newcommand{\Id}{\mathbbm{1}}

% The list of PHY.tex commands
\newcommand{\received}[1]{\hat{#1}}
\newcommand{\proba}[1]{\mathbb{P}_{#1}}
\newcommand{\real}[1]{\mathbb{R}^{#1}}
\newcommand{\dmin}{d_{\min}}
\newcommand{\decisionf}[1]{g(#1)}

\newcommand{\energy}{\mathcal{E}}
\newcommand{\power}{P}

\newcommand{\decfunc}{g}
\newcommand{\decreg}{D}
\renewcommand{\vec}{\bm}
\newcommand{\px}[1]{p_{\vec{x}}(#1)}
\newcommand{\ps}[1]{p_{\vec{s}}(#1)}



%\newcommand{\rate}{R}
\newcommand{\brate}{R_b}
\newcommand{\srate}{R_s}
\newcommand{\Tsym}{T_s}
\newcommand{\Tbit}{T_b}
\newcommand{\N}{N}
\newcommand{\M}{M}
\newcommand{\sign}{\text{sign}}


\def\MYdef{\mathrel{\stackrel{\rm def}=}}

\newcommand{\Real}[1]{\Re\left\{#1\right\}}
\newcommand{\Imag}[1]{\Im\left\{#1\right\}}

\newcommand{\Energy}{\mathcal{E}}
\newcommand{\Const}{\mathfrak{A}}

% Различные преобразования
\newcommand{\Freq}{f}
\newcommand{\Complex}[1]{\tilde{#1}}
\newcommand{\Hilbert}[1]{\hat{#1}}
\newcommand{\fourier}[1]{\mathcal{F}[#1](\Freq)}
\newcommand{\bfourier}[1]{\mathcal{F}^{-1}[#1](t)}
\newcommand{\Lower}[2]{\mathcal{L}_{#2}[#1](t)}

\newcommand{\fupp}{f_{u}}
\newcommand{\fdis}{f_{d}}

\newcommand{\DeltaFreq}{\Delta \Freq}
\newcommand{\DeltaTime}{\Delta t}

\newcommand{\NumberOfSubcarriers}{N_c}
\newcommand{\Nc}{\NumberOfSubcarriers}
\newcommand{\Nfft}{N}
\newcommand{\FreqIndex}{n}
\newcommand{\FI}{\FreqIndex}
\newcommand{\TimeIndex}{k}
\newcommand{\TI}{\TimeIndex}

\newcommand{\OfdmSubcarrier}[1]{\ifthenelse{\isempty{#1}}{\Freq_\FreqIndex}{\Freq_{#1}}}
\newcommand{\OfdmSubcarriers}{\{\Freq_\FreqIndex\}_{\FreqIndex=0}^{\NumberOfSubcarriers-1}}

\newcommand{\TxSignal}{x}
\newcommand{\OfdmSignal}{\TxSignal}
\newcommand{\RealOfdmSignal}{\OfdmSignal}
\newcommand{\CompOfdmSignal}{\tilde{\TxSignal}}

\newcommand{\OfdmSubsignal}[1]{\ifthenelse{\isempty{#1}}{\OfdmSignal_{\FreqIndex}(t)}{\OfdmSignal_{#1}(t)}}
\newcommand{\RealOfdmSubsignal}[1]{\OfdmSubsignal{#1}}
\newcommand{\CompOfdmSubsignal}[1]{\ifthenelse{\isempty{#1}}{\CompOfdmSignal_{\FreqIndex}(t)}{\CompOfdmSignal_{#1}(t)}}

\newcommand{\OfdmSubsignals}{\{\OfdmSignal_{\FreqIndex}(t)\}_{\FreqIndex=0}^{\NumberOfSubcarriers-1}}
\newcommand{\RealOfdmSubsignals}{\OfdmSubsignals}
\newcommand{\CompOfdmSubsignals}{\{\CompOfdmSignal_{\FreqIndex}(t)\}_{\FreqIndex=0}^{\NumberOfSubcarriers-1}}

\newcommand{\pulse}{g}
\newcommand{\symb}{A}


% The list of PROBLEMS.tex commands

% LTE
\newcommand{\BSs}{\mathbb{BS}}
\newcommand{\UEs}{\mathbb{MS}}
\newcommand{\BWs}{\mathbb{BW}}
\newcommand{\NoBS}{\mathcal{B}}
\newcommand{\NoUE}{\mathcal{U}}
\newcommand{\NoBW}{\mathcal{W}}

\newcommand{\Tres}{T_{res}}
\newcommand{\Tin}{T_{in}}
\newcommand{\tres}{t_{res}}
\newcommand{\tin}{t_{in}}
\newcommand{\age}{h}
\newcommand{\recv}[1]{S_{#1}}
\newcommand{\PER}{p}

\newcommand{\BlockSize}{B}
\newcommand{\ActiveBlockSize}{N}
\newcommand{\QueueSize}{Q}

\newcommand{\packet}{\boldsymbol{c}}
\newcommand{\nativepackets}{\boldsymbol{M}}
\newcommand{\allpackets}{\boldsymbol{X}}
\newcommand{\rate}{r}
\newcommand{\group}[1]{\mathcal{G}_{#1}}

\newcommand{\weight}{w}
\newcommand{\prob}{\mathbb{P}}

%Simulink
\newcommand{\fcn}[1]{\texttt{#1}}
\newcommand{\variable}[1]{\texttt{#1}}
\newcommand{\question}[1]{\textbf{#1}}
\newcommand{\command}[1]{\texttt{#1}}

%\makeatletter
%\def\eprob{\@ifnextchar[{\@condeprob}{\@notcondeprob}}
%\def\@notcondeprob{\mathbb{P}_{e}}
%\def\@condeprob[#1]{\mathbb{P}_{e|#1}}
%\makeatother

%\makeatletter
%\def\prob{\@ifnextchar[{\@condprob}{\@notcondprob}}
%\def\@condprob[#1]#2{\mathbb{P}\{#2|#1\}}
%\def\@notcondprob#1{\mathbb{P}\{#1\}}
%\makeatother


% Neural network
% Convolutional Networks

\newcommand{\partder}[2]{\frac{\partial #1 }{\partial #2}}

\newcommand{\Loss}{\mathcal{L}}

\newcommand{\Input}{I}
\newcommand{\Output}{O}

\newcommand{\Weights}{W}
\newcommand{\Biases}{\boldsymbol{b}}
\newcommand{\Bias}{b}
\newcommand{\Filter}{F}

\newcommand{\FilterAnchor}{\mathbb{d}}
\newcommand{\AffectField}{A}
\newcommand{\ReceptiveField}{R}
\newcommand{\FilterSize}{\boldsymbol{f}}

\newcommand{\StrideSize}{\boldsymbol{s}}
\newcommand{\StrideSizeH}{\StrideSize_h}
\newcommand{\StrideSizeW}{\StrideSize_w}

\newcommand{\NumOfSamples}{\mathcal{N}}
\newcommand{\NumOfDims}{\mathcal{D}}
\newcommand{\NumOfChannels}{\mathcal{C}}
\newcommand{\NumOfFilters}{\mathcal{F}}
\newcommand{\HiddenSize}{\mathcal{H}}

\newcommand{\bolda}{\boldsymbol{a}}
\newcommand{\boldb}{\boldsymbol{b}}
\newcommand{\bolde}{\boldsymbol{e}}
\newcommand{\boldw}{\boldsymbol{w}}
\newcommand{\boldx}{\boldsymbol{x}}
\newcommand{\boldy}{\boldsymbol{y}}
\newcommand{\boldz}{\boldsymbol{z}}

\newcommand{\boldA}{\boldsymbol{A}}
\newcommand{\boldB}{\boldsymbol{B}}
\newcommand{\boldC}{\boldsymbol{C}}
\newcommand{\boldE}{\boldsymbol{E}}
\newcommand{\boldI}{\boldsymbol{I}}
\newcommand{\boldW}{\boldsymbol{W}}
\newcommand{\boldX}{\boldsymbol{X}}
\newcommand{\boldY}{\boldsymbol{Y}}
\newcommand{\boldZ}{\boldsymbol{Z}}

\newcommand{\boldbeta}{\boldsymbol{\beta}}
\newcommand{\boldmu}{\boldsymbol{\mu}}
\newcommand{\boldsigma}{\boldsymbol{\sigma}}

\newcommand{\eps}{\varepsilon}

\chapter{Полносвязные Сети}

\begin{itemize}
	\item $\NumOfSamples$ --- размер батча
	\item $\NumOfDims$ --- размерность вектора признаков
\end{itemize}

\section{Dense Layer}

\paragraph{Векторный вид}

\begin{gather}
\boldy^T = \boldx^T \boldW + \boldb^T \\
\boldy = \boldW^T \boldx + \boldb
\end{gather}

\begin{gather}
\partder{\Loss}{\boldx} = \partder{\boldy}{\boldx} \partder{\Loss}{\boldy} = \boldW \partder{\Loss}{\boldy}
\end{gather}

\begin{gather*}
\partder{\Loss}{\boldw_i} = \partder{\boldy}{\boldw_i} \partder{\Loss}{\boldy} = \partder{y_i}{\boldw_i} \partder{\Loss}{y_i} = \boldx \cdot \partder{\Loss}{y_i} \Rightarrow \partder{\Loss}{\boldW} = \boldx \left(\partder{\Loss}{\boldy}\right)^T
\end{gather*}

\begin{gather*}
\partder{\Loss}{\boldb} = \partder{\boldy}{\boldb} \partder{\Loss}{\boldy} = \partder{\Loss}{\boldy}.
\end{gather*}


\paragraph{Матричный вид}

\begin{gather*}
\boldY = \boldX \boldW + \boldB, 
\end{gather*}
где 
$$\boldB = [\underbrace{\boldb, \boldb, \dots, \boldb}_{\NumOfSamples}]^T \in \mathbb{R}^{\NumOfSamples \times \NumOfDims}.$$
В таком случае 
\begin{gather*}
\partder{\Loss}{\boldX} = [\partder{\Loss}{\boldx_1}, \dots, \partder{\Loss}{\boldx_\NumOfSamples}]^T = [\boldW  \partder{\Loss}{\boldy_1}, \dots,  \boldW \partder{\Loss}{\boldy_{\NumOfSamples}}]^T =  [\partder{\Loss}{\boldy_1}, \dots,  \partder{\Loss}{\boldy_{\NumOfSamples}}]^T \boldW^T =  \partder{\Loss}{\boldY} \boldW^T.
\end{gather*}

\begin{gather*}
\partder{\Loss}{\boldw_i} = \Sum_{k=1}^{\NumOfSamples} \partder{y_i^{(k)}}{\boldw_i} \partder{\Loss}{y_i^{(k)}} = \Sum_{k=1}^{\NumOfSamples} \boldx^{(k)} \partder{\Loss}{y_i^{(k)}} \Rightarrow \partder{\Loss}{\boldW} = \Sum_{k=1}^{\NumOfSamples} \boldx^{(k)} \left(\partder{\Loss}{\boldy^{(k)}}\right)^T = \boldX^T \partder{\Loss}{\boldY}.
\end{gather*}

\begin{gather*}
\partder{\Loss}{\boldb} = \Sum_{k=1}^{\NumOfSamples} \partder{\boldy^{(k)}}{\boldb} \partder{\Loss}{\boldy^{(k)}} = \Sum_{k=1}^{\NumOfSamples} \partder{\Loss}{\boldy^{(k)}}
\end{gather*}

\section{SoftMax}

\subsection{Forward propagation}

\subsection{Backward propagation}

\begin{gather*}
\partder{\Loss}{\boldx} = \partder{\boldy}{\boldx} \partder{\Loss}{\boldy}
\end{gather*}

\begin{gather*}
\partder{y_i}{x_j} = \partial \left(\frac{e^{x_j}}{\Sum_d e^{x_d}}\right) / \partial x_j = \frac{1}{\Sum_{d=1}^{\NumOfDims} e^{x_d}} \partder{e^{x_i}}{x_j} - \frac{e^{x_i}}{\left(\Sum_{d=1}^{\NumOfDims} e^{x_d}\right)^2} \partder{\left(\Sum_{d=1}^{\NumOfDims} e^{x_d}\right)}{x_j} = \frac{e^{x_i}\delta_{ij}}{\Sum_{d=1}^{\NumOfDims} e^{x_d}} - \frac{e^{x_i}}{\Sum_{d=1}^{\NumOfDims} e^{x_d}} \cdot \frac{e^{x_j}}{\Sum_{d=1}^{\NumOfDims} e^{x_d}} = y_j \delta_{ij} - y_i y_j.
\end{gather*}
Таким образом,
\begin{gather*}
\partder{\boldy}{\boldx} = \diag[y_1, y_2, \dots, y_\NumOfDims] - \boldy \boldy^T
\end{gather*}
\begin{gather*}
\partder{\Loss}{\boldx} = \partder{\boldy}{\boldx} \partder{\Loss}{\boldy} = \left(\diag[y_1, y_2, \dots, y_\NumOfDims] - \boldy \boldy^T\right) \partder{\Loss}{\boldy} = \diag[y_1, y_2, \dots, y_\NumOfDims] \partder{\Loss}{\boldy} - \boldy \cdot \left\langle \boldy, \partder{\Loss}{\boldy} \right\rangle = \\
= \boldy \otimes \partder{\Loss}{\boldy} - \boldy \left\langle \boldy, \partder{\Loss}{\boldy} \right\rangle
\end{gather*}

\section{Batch Normalization}

Пусть $\boldx$ --- это вектор на входе слоя ($x \in \mathbb{R}^{\NumOfSamples}$). Тогда вектор $\boldy$ на выходе слоя есть
$$
\boldy = \gamma \cdot \frac{\boldx - \mu}{\sqrt{\sigma^2 + \eps}} + \boldbeta,
$$
где $$\mu = \frac{1}{\NumOfSamples}\Sum_{i=1}^\NumOfSamples x_i, \qquad \sigma^2 = \frac{1}{\NumOfSamples}\Sum_{i = 1}^\NumOfSamples (x_i - \mu)^2 = \frac{1}{\NumOfSamples} x_i^2 - \left(\frac{1}{\NumOfSamples} x_i \right)^2.$$

Пусть дана производная функции потерь по $\boldy$, т.е. $\partial \Loss / \partial \boldy$. Найдем производные функции потерь по $\boldx$, $\boldbeta$, $\gamma$.
Сначала найдем $\partial \Loss / \partial \boldx$:
\begin{gather}
\partder{\Loss}{\boldx} = \underbrace{\partder{\boldy}{\boldx}}_{\mathbb{R}^{\NumOfSamples \times \NumOfSamples}} \cdot \underbrace{\frac{\Loss}{\boldy}}_{\mathbb{R}^{\NumOfSamples}},
\end{gather}
где 
\begin{equation}
\partder{\boldy}{\boldx} = 
\begin{pmatrix}
\partder{y_1}{x_1} &\partder{y_2}{x_1} &\dots &\partder{y_\NumOfSamples}{x_1} \\
\partder{y_1}{x_2} &\partder{y_2}{x_2} &\dots &\partder{y_\NumOfSamples}{x_2} \\
\vdots 			   &\vdots 			   &\ddots &\vdots                        \\
\partder{y_1}{x_\NumOfSamples} &\partder{y_2}{x_\NumOfSamples} &\dots &\partder{y_\NumOfSamples}{x_\NumOfSamples} \\
\end{pmatrix}
\end{equation}

\begin{gather*}
\partder{\boldy}{\boldx} = \gamma \partder{\left(\frac{\boldx - \mu}{\sqrt{\sigma^2 + \eps}}\right)}{\boldx}.
\end{gather*}
Рассмотрим производную 
\begin{gather*}
\partder{y_i}{x_j} = \gamma \partial\left(\frac{x_i - \mu}{\sqrt{\sigma^2 + \eps}}\right)/\partial x_j = 
\gamma \partder{(x_i - \mu)}{x_j} \frac{1}{\sqrt{\sigma^2 + \eps}} - \frac{\gamma}{2} \frac{x_i - \mu}{(\sigma^2 + \eps)^{3/2}} \partder{(\sigma^2 + \eps)}{x_j} = \\
= \gamma \left(\delta_{ij} - \frac{1}{\NumOfSamples}\right) \frac{1}{\sqrt{\sigma^2 + \eps}} - \frac{\gamma}{2} \frac{x_i - \mu}{(\sigma^2 + \eps)^{3/2}} \cdot 2 \left(\frac{x_j}{\NumOfSamples} -  \frac{\mu}{\NumOfSamples} \right) =  \gamma \left(\delta_{ij} - \frac{1}{\NumOfSamples}\right) \frac{1}{\sqrt{\sigma^2 + \eps}} - \gamma \frac{(x_i - \mu)(x_j - \mu)}{\NumOfSamples(\sigma^2 + \eps)^{3/2}}.
\end{gather*}
Таким образом
\begin{gather*}
\partder{\boldy}{\boldx} = \frac{\gamma}{\sqrt{\sigma^2 + \eps}} \left(\left(\boldI - \frac{1}{\NumOfSamples}\boldE\right) -  \frac{\boldC}{\sigma^2 + \eps}\right),
\end{gather*}
где $\boldC = \frac{1}{\NumOfSamples}(\boldx - \boldmu)(\boldx - \boldmu)^T$, $\boldI$ --- единичная матрица размера $\NumOfSamples \times \NumOfSamples$, $\boldE$ --- матрица, полностью состящая из единиц, размера $\NumOfSamples \times \NumOfSamples$. Тогда
\begin{gather*}
\partder{\Loss}{\boldx} = \frac{\gamma}{\sqrt{\sigma^2 + \eps}} \left(  \left(\boldI - \frac{1}{\NumOfSamples}\boldE\right) -  \frac{\boldC}{\sigma^2 + \eps}\right) \partder{\Loss}{\boldy} = 
\frac{\gamma}{\sqrt{\sigma^2 + \eps}} \left(\partder{\Loss}{\boldy} - \bolde \cdot \overline{\partder{\Loss}{\boldy}} - \frac{\boldx - \mu}{\NumOfSamples(\sigma^2 + \eps)} \cdot \langle \boldx - \mu, \frac{\Loss}{\boldy}\rangle \right),
\end{gather*}
где $\bolde$ --- столбец из единиц размерности $\NumOfSamples$, $\overline{\partder{\Loss}{\boldy}}$ --- среднее значение элементов вектора $\partder{\Loss}{\boldy}$. Обозначив через $\boldz$ ``стандартизованный'' вектор $\boldx$ ($\boldz = (\boldx - \mu) / \sigma$), получим
\begin{gather*}
\partder{\Loss}{\boldx} =\frac{\gamma}{\sqrt{\sigma^2 + \eps}} \left(\partder{\Loss}{\boldy} - \bolde \cdot \overline{\partder{\Loss}{\boldy}} - \frac{\boldx - \mu}{\NumOfSamples(\sigma^2 + \eps)} \cdot \langle \boldx - \mu, \frac{\Loss}{\boldy}\rangle \right) = 
\frac{\gamma}{\sqrt{\sigma^2 + \eps}} \left(\partder{\Loss}{\boldy} - \bolde \cdot \overline{\partder{\Loss}{\boldy}}  - \frac{\boldz}{\NumOfSamples} \cdot \langle \boldz, \frac{\Loss}{\boldy}\rangle \right).
\end{gather*}

Теперь найдем производные $\partial \Loss / \partial \boldbeta$ и $\partial \Loss / \partial \gamma$:
\begin{gather*}
\partder{\Loss}{\boldbeta} = \partder{\boldy}{\boldbeta} \partder{\Loss}{\boldy} = \boldI \partder{\Loss}{\boldy} = \partder{\Loss}{\boldy}, \\
\partder{\Loss}{\gamma} = \left(\partder{\boldy}{\gamma}\right)^T \partder{\Loss}{\boldy} = \frac{\boldx - \mu}{\sqrt{\sigma + \eps}}
\end{gather*}

\subsection{Реализация в Python}

Представленная ниже реализация предполагает, что 
\begin{itemize}
	\item $\boldX \in \mathbb{R}^{\NumOfSamples \times \NumOfDims}$ --- матрица объектов-признаков
	\item $\boldZ \in \mathbb{R}^{\NumOfSamples \times \NumOfDims}$ --- ``стандартизованная'' вдоль оси 0 матрица $\boldX$, т.е. выход слоя \texttt{BatchNormalization}
	\item $\boldmu \in \mathbb{R}^{\NumOfSamples}$ --- средние значения для признаков
	\item $\boldsigma \in \mathbb{R}^{\NumOfSamples}$ --- стандартные отклонения для признаков
\end{itemize}

\begin{python}
def batchnorm_backward_alt(output_grad, cache):
  X, Z, mu, sigma, gamma, beta, eps = cache
  N, D = X.shape
  X_grad = gamma / np.sqrt(sample_var + eps)[None, :] *\ 
           ((output_grad - np.mean(output_grad, axis=0)[None, :]) -\
           Z * np.mean(np.multiply(Z, output_grad), axis=0)[None, :]) 
  beta_grad = np.sum(output_grad, axis=0)
  gamma_grad = np.sum(np.multiply(X_norm, output_grad), axis=0)
  return X_grad, gamma_grad, beta_grad
\end{python}

\section{MulticlassLogLoss}
На вход данного уровня поступает вектор $\boldp \in \mathbb{R}^{\NumOfDims}$ или же в случае использования батчей матрица $\boldP \in \mathbb{R}^{\NumOfSamples \times \NumOfDims}$. Данный уровень обычно является последним уровнем нейронной сети и используется для оценки вероятностей принадлежности объектов к классам. Поэтому далее будем считать, что $\NumOfDims = \NumOfClasses$, где $\NumOfClasses$ --- количество классов в выборке.

\begin{gather*}
\Loss = \frac{1}{\NumOfSamples} \Sum_{n = 1}^{\NumOfSamples} \Sum_{k = 1}^{\NumOfClasses} y_k^{(n)} \log p_n^{(k)} = \frac{1}{\NumOfSamples} \Sum_{n=1}^{\NumOfSamples} \Loss_n.
\end{gather*}

\begin{gather*}
\partder{\Loss}{p_n^{(k)}} = \frac{1}{\NumOfSamples} \frac{y_n^{(k)}}{p_n^{(k)}}
\end{gather*}
\begin{gather*}
\partder{\Loss}{\boldp_n} = [0, \dots,  \frac{1}{\NumOfSamples} \frac{y_n^{(k_n^*)}}{p_n^{(k_n*)}}, \dots, 0], 
\end{gather*}
где $k_n^*$ --- это истинный класс объекта $n$. Таким образом $\partder{\Loss}{\boldP}$ --- это матрица размера $\NumOfSamples \times \NumOfClasses$, имеющая вид
\begin{gather*}
\partder{\Loss}{\boldP} = \frac{1}{\NumOfSamples}
\begin{bmatrix}
\frac{y_{1, k^*_1}}{p_{1, k^*_1}} &\dots  &0 &\dots  &0 \\
0 &\dots  &\frac{y_{2, k^*_2}}{p_{2, k^*_2}} &\dots  &0 \\
\vdots &\vdots  &\vdots &\ddots  &\vdots \\
0 &\dots  &0 &\dots  &\frac{y_{\NumOfSamples, k^*_\NumOfSamples}}{p_{\NumOfSamples, k^*_\NumOfSamples}} \\
\end{bmatrix}
\end{gather*}

\chapter{Сверточные Сети}

\section{Обратное распространение ошибки}

\subsection{Случай $\text{stride} = 1$}

\begin{equation}
\partder{\Loss}{\Input[c, h, w]}= \Sum_{(i, j) \in \AffectField(h, w)} 
\partder{\Output[c, i, j]}{\Input[c, h, w]} \cdot \partder{\Loss}{\Output[c, i, j]}
\end{equation}

Введем вспомогательные термины. 

\paragraph{Область действия.}
Рассмотрим элемент $\Input[h, w]$ входа сверточного слоя ($h \in [0, H - 1]$, $w \in [0, W - 1]$). \textbf{Областью действия} $\AffectField(h, w)$ элемента $\Input[h, w]$ называется множество координат элементов матрицы $\Output$, которые зависят от элемента $I[h, w]$:
\begin{equation}
\AffectField(h, w) \triangleq \{(i, j)\colon i \in \AffectField(h), j\in \AffectField(w)\},
\end{equation}
где
\begin{align}
\AffectField(h)& = \{i \colon \max\{0, h - \FilterSize[h]\} \le i \le \min\{h, H - \FilterSize[h]\}\}, \\
\AffectField(w)& = \{j \colon \min\{0, w - \FilterSize[w]\} \le j \le \max\{w, W - \FilterSize[w]\}\}.
\end{align}

\paragraph{Область восприимчивости.}
Теперь рассмотрим элемент $\Output[i, j]$ выхода сверточного слоя ($i \in [0, H - \FilterSize_h]$, $j \in [0, W - \FilterSize_w]$). \textbf{Областью восприимчивости} $\ReceptiveField(h, w)$ элемента $\Output[i, j]$ называется множество координат элементов матрицы $\Input$, от которых зависит значение элемента $\Output[i, j]$:
\begin{equation}
\label{eq:receptive_field}
\ReceptiveField(i, j) = \{(h, w) \colon h \in \ReceptiveField(i), w \in \ReceptiveField(j)\},
\end{equation}
где
\begin{align}
\label{eq:receptive_field:add}
\ReceptiveField(i) &= \{h \colon i \le h \le \min\{i + \FilterSize_h - 1, H - 1\}\}, \\
\ReceptiveField(j) &= \{w \colon j \le w \le \min\{j + \FilterSize_w - 1, W - 1\}\}.
\end{align}


\begin{equation}
\label{eq:conv:backprop:1}
\partder{\Loss}{\Input[c, h, w]}= \Sum_{(i, j) \in \AffectField(h, w)} 
\partder{\Output[c, i, j]}{\Input[c, h, w]} \cdot \partder{\Loss}{\Output[c, i, j]} = 
\Sum_{(i, j) \in \AffectField(h, w)} W[c, h - i, w - j] \partder{\Loss}{\Output[c, i, j]}
\end{equation}
Уже здесь можно заметить, что нахождение производных по $\Input$ напоминает свертку производных по выходу $\partder{\Loss}{\Output}$
с транспонированной (вдоль второго и третьего измерений) матрицей весов $W$.

Рассмотрим пока для определенности только точки $(h, w)$ из диапазона $(h, w) \in [\FilterSize[h] - 1, H - \FilterSize[h]] \times [\FilterSize[w] - 1, W - \FilterSize[w]]$. В этом случае сумму из~\eqref{eq:conv:backprop:1} можно записать в виде
\begin{equation}
\label{eq:conv:backprop:2}
\partder{\Loss}{\Input[c, h, w]} = \Sum_{\substack{i \in [h - \FilterSize[h] + 1, h] \\ j \in [w - \FilterSize[w] + 1, w]}} W[c, h - i, w - j] \partder{\Loss}{\Output[c, i, j]},
\end{equation}
не содержащим граничных условий на $i$ и $j$. Далее введем транспонированную вдоль второго и третьего измерений матрицу весов $W'$:
\begin{equation}
W'[i, j] = W[c, \FilterSize[h] - 1 - i, \FilterSize[w] - 1 - j], \quad c \in [0, \NumOfFilters - 1], i \in [0, \FilterSize[h] - 1], j \in [0, \FilterSize[w] - 1].
\end{equation}
В таком случае~\eqref{eq:conv:backprop:2} можно далее записать в виде
\begin{equation}
\partder{\Loss}{\Input[c, h, w]} = \Sum_{\substack{i \in [h - \FilterSize[h] + 1, h] \\ j \in [w - \FilterSize[w] + 1, w]}} W'[c, i, j] \partder{\Loss}{\Output[c, i, j]}.
\end{equation}

Можно в принципе избавиться от граничных условий и записать для любых $(h, w) \in [0, H - 1] \times 0, W - 1]$, если положить, что значения по индексам, отсутствующим в матрице $\partial \Loss / \partial I$, равны нулю:
\begin{equation}
\partder{\Loss}{\Input[c, h, w]} = \Sum_{\substack{i \in [h - \FilterSize[h] + 1, h] \\ j \in [w - \FilterSize[w] + 1, w]}} W'[c, i, j] \partder{\Loss}{\Output[c, i, j]}.
\end{equation}
Фактически это эквивалентно тому, что матрица $\partder{\Loss}{\Output[c, i, j]}$ дополняется $\FilterSize[h] - 1$ вдоль второго измерения и $\FilterSize[w] - 1$ вдоль третьего измерения с каждой стороны. Полученная матрица имеет размер $(H + \FilterSize[h] - 1) \times (W + \FilterSize[w] - 1)$. 

\subsection{Общий случай $\text{stride} \ge 1$}
В данном случае размер матрицы $\Input$ как и ранее есть $H \times W$. В общем случае $\StrideSize_h \ge 1$ и/или $\StrideSize_w \ge 1$ для размеры $H' $ и $W'$ матрицы $\Output$ равны
\begin{equation}
H' = \frac{H - \FilterSize_h}{\StrideSize_h} + 1, \qquad W' = \frac{W - \FilterSize_w}{\StrideSize_w} + 1.
\end{equation}
Здесь как и ранее полагаем, что значения $H$, $W$, $\StrideSize$, $\FilterSize$ согласованы путем, например, предварительного дополнения входа сверточного слоя нулями, так что $H'$  и $W'$ --- целые числа.

В данном случае требуется скорректировать ранее введенные области действия и восприимчивости.

\paragraph{Область восприимчивости $\ReceptiveField(i, j)$}
Рассмотрим элемент $\Output[i, j]$ ($i \in [0, H'- 1]$, $j \in [0, W' - 1]$). Этот элемент зависит от элементов матрицы $\Input$, расположенных по следующим координатам $\{(h, w)\}$:
\begin{gather*}
h \in [i \cdot \StrideSize_h, i \cdot \StrideSize_h + \FilterSize_h - 1], \\
w \in [j \cdot \StrideSize_w, j \cdot \StrideSize_w + \FilterSize_w - 1].
\end{gather*}
Заметим, что при всех допустимых $i \in [0, H'- 1]$ и $j \in [0, W' - 1]$ значения $h$ и $w$ лежат в допустимых множествах $[0, H - 1]$ и $[0, W - 1]$. Таким образом, область восприимчивости имеет вид:
\begin{equation}
\ReceptiveField(i, j) = \{(h, w)\colon h \in [i \cdot \StrideSize_h, i \cdot \StrideSize_h + \FilterSize_h - 1], w \in [j \cdot \StrideSize_w, j \cdot \StrideSize_w + \FilterSize_w - 1]\}, \quad i \in [0, H'- 1], j \in [0, W' - 1].
\end{equation}

\paragraph{Область влияния (действия) $\AffectField(h, w)$.}
Рассмотрим, на какие элементы матрицы $\Output$ влияет элемент $\Input[h, w]$. Чтобы элемент $\Input[h, w]$ влиял на $\Output[i, j]$, нужно, чтобы $(h, w) \in \ReceptiveField(i, j)$, т.е.
\begin{equation}
i \cdot \StrideSize_h \le h \le i \cdot \StrideSize_h + \FilterSize_h - 1, \qquad j \cdot \StrideSize_w \le w \le j \cdot \StrideSize_w + \FilterSize_w - 1.
\end{equation}
Осталось лишь определить все те пары $(i, j)$, для которых данное условие выполнено. 
Из условий выше следует, что
\begin{equation}
\frac{h - \FilterSize_h + 1}{\StrideSize_h} \le i \le \frac{h}{\StrideSize_h}, \qquad \frac{w - \FilterSize_w + 1}{\StrideSize_w} \le j \le \frac{w}{\StrideSize_w}. 
\end{equation}
С граничными условиями, данные неравенства для определения области влияния $\AffectField(h, w)$ переходят в

С учетом граничных условий, области влияния $\AffectField(h)$ и $\AffectField(w)$ вдоль высоты и ширины определяются как
\begin{gather*}
\AffectField(h; \StrideSize_h) = \left\{i \in \mathbb{Z} \colon \max\{\frac{h - \FilterSize_h + 1}{\StrideSize_h}, 0\} \le i \le \min\{\frac{h}{\StrideSize_h}, \frac{H - \FilterSize_h}{\StrideSize_h}\}\right\},\\
\AffectField(w; \StrideSize_w) = \left\{j \in \mathbb{Z} \colon \max\{\frac{w - \FilterSize_w + 1}{\StrideSize_w}, 0\} \le j \le \min\{\frac{w}{\StrideSize_w}, \frac{W - \FilterSize_w}{\StrideSize_w}\}\right\}.\\
\end{gather*}
%Граничные условия активируются при следующих значениях $h$ и $w$ соответственно
%\begin{gather*}
%h \in [0, \FilterSize_h - 1) \cup [H - \FilterSize_h + 1, H),\\
%w \in [0, \FilterSize_w - 1) \cup [W - \FilterSize_w + 1, W).
%\end{gather*}

Пусть $h = q_h\StrideSize_h + r_h$, $q_h \in \mathbb{Z}$, $r_h \in [0, \StrideSize_h - 1]$.
\begin{equation*}
\frac{h - \FilterSize_h + 1}{\StrideSize_h} = q_h + \frac{r_h}{\StrideSize_h} -\frac{\FilterSize_h - 1}{\StrideSize_h} \le i \le q_h + \frac{r_h}{\StrideSize_h} = \frac{h}{\StrideSize_h}.
\end{equation*}

\subsubsection{Производные функции потерь $\Loss$}
\paragraph{Производная по входу $\Input$.}
%\substack{i \in \AffectField(h; \StrideSize_h) \\ j \in \AffectField(w; \StrideSize_w)}}
\begin{equation}
\partder{\Loss}{\Input[h, w]}= \Sum_{(i, j) \in \AffectField(h, w; \StrideSize)} 
\partder{\Output[i, j]}{\Input[h, w]} \cdot \partder{\Loss}{\Output[i, j]} =
\Sum_{(i, j) \in \AffectField(h, w; \StrideSize)} \Weights[h - i \cdot \StrideSize_h, w - j \cdot \StrideSize_w] \cdot \partder{\Loss}{\Output[i, j]}.
\end{equation}
Так как множество $\AffectField(h, w; \StrideSize)$ уже найдено, то поточечное вычисление данной производной не представляет проблем.

В случае наличия нескольких каналов производная по входу $\Input[c, h, w]$ вычисляется следующим образом:
\begin{equation}
\partder{\Loss}{\Input[c, h, w]}= \Sum_{l=0}^{\NumOfChannels' - 1}\Sum_{(i, j) \in \AffectField(h, w; \StrideSize)} 
\partder{\Output[l, i, j]}{\Input[c, h, w]} \cdot \partder{\Loss}{\Output[i, j]} =
\Sum_{(i, j) \in \AffectField(h, w; \StrideSize)} \Weights[c, h - i \cdot \StrideSize_h, w - j \cdot \StrideSize_w] \cdot \partder{\Loss}{\Output[l, i, j]}.
\end{equation}

\paragraph{Производная по весам $\Weights$.}
В случае одного входного и выходного каналов производная по весу имеет вид:
\begin{equation}
\partder{\Loss}{\Weights[h, w]} = \Sum_{\substack{0 \le i' < H' \\ 0 \le j' < W'}} \partder{\Output[i', j']}{\Weights[h, w]} \cdot  \partder{\Loss}{\Output[i', j']} =  \Sum_{\substack{0 \le i' < H' \\ 0 \le j' < W'}} \Input[i' \cdot \StrideSizeH + h, j' \cdot \StrideSizeW + w] \cdot \partder{\Loss}{\Output[i', j']}.
\end{equation}

В случае наличия $\NumOfChannels\ge 1$ входных каналов и одного выходного производная принимает вид
\begin{equation}
\partder{\Loss}{\Weights[c, h, w]} = \Sum_{\substack{0 \le i' < H' \\ 0 \le j' < W'}} \partder{\Output[i', j']}{\Weights[c, h, w]} \cdot  \partder{\Loss}{\Output[i', j']} =  \Sum_{\substack{0 \le i' < H' \\ 0 \le j' < W'}} \Input[c, i' \cdot \StrideSizeH + h, j' \cdot \StrideSizeW + w] \cdot \partder{\Loss}{\Output[i', j']}.
\end{equation}

В наиболее общем случае $\NumOfChannels\ge 1$ входных каналов и одного выходного производная принимает вид
\begin{gather}
\partder{\Loss}{\Weights[c', c, h, w]} = \Sum_{\substack{0 \le i' < H' \\ 0 \le j' < W'}} \partder{\Output[c', i', j']}{\Weights[c', c, h, w]} \cdot  \partder{\Loss}{\Output[c', i', j']} = \Sum_{\substack{0 \le i' < H' \\ 0 \le j' < W'}} \Input[c, i' \cdot \StrideSizeH + h, j' \cdot \StrideSizeW + w] \cdot \partder{\Loss}{\Output[c', i', j']}.
\end{gather}

\paragraph{Производная по смещению $\Biases$.}
Каждому каналу на выходе соответствует свое значение смещения.

\begin{equation}
\partder{\Loss}{\Bias} = \Sum_{\substack{0 \le i' < H' \\ 0 \le j' < W'}} \partder{\Output[i', j']}{\Bias} \cdot \partder{\Loss}{\Output[i', j']} = \Sum_{\substack{0 \le i' < H' \\ 0 \le j' < W'}} \partder{\Loss}{\Output[i', j']}.
\end{equation}

\begin{equation}
\partder{\Loss}{\Biases[c']} = \Sum_{\substack{0 \le i' < H' \\ 0 \le j' < W'}} \partder{\Output[c', i', j']}{\Biases[c']} \cdot \partder{\Loss}{\Output[c', i', j']} = \Sum_{\substack{0 \le i' < H' \\ 0 \le j' < W'}} \partder{\Loss}{\Output[c', i', j']}.
\end{equation}

\subsection{Альтернативная реализация}
Пусть 
\begin{itemize}
	\item $\boldX$ --- входной тензор сверточного слоя
	\item $\boldY$ --- тензор на выходе сверточного слоя
	\item Свертка считается в режиме \texttt{VALID}
\end{itemize}

Для начала рассмотрим случай двумерного входного тензора, т.е.
\begin{itemize}
	\item $\boldX$ имеет размер $\InputH \times \InputW$
	\item $\boldY$ имеет размер $\OutputH \times \OutputW$
\end{itemize}

Пусть сначала считается полная свертка с $\StrideSizeH = \StrideSizeW = 1$, тогда на выходе сначала получим тензор $\boldZ$ размера $(\InputH - \FilterSizeH + 1) \times (\InputW - \FilterSizeW + 1)$. Однако не все его значения нам нужны


\section{Слой Max Pooling}

\subsection{Прямое распространение}

\subsection{Обратное распространение}

\chapter{Рекуррентные сети (Recurrent neural networks)}


\begin{equation*}
\boldy = f(\boldV \boldh + \boldW \boldx + \boldb)
\end{equation*}

\begin{eqnarray*}
\boldh_1 &= &f(\boldV \boldh_0 + \boldW \boldx_0 + \boldb)\\
\boldh_2 &= &f(\boldV \boldh_1 + \boldW \boldx_1 + \boldb)\\
\boldh_3 &= &f(\boldV \boldh_2 + \boldW \boldx_2 + \boldb)\\
\dots \\
\boldh_k &= &f(\boldV \boldh_{k-1} + \boldW \boldx_{k-1} + \boldb)\\
\end{eqnarray*}

Если формулы прямого распространения не вызывают особых сложностей, то обратное распространение требует значительно большего количества выкладок.

\section{Обратное распространение ошибки}

$$
\boldy = \boldW \boldx + \boldb
$$
\begin{table}
	\centering
	\begin{tabular}{lllclll}
		$\frac{d \Loss}{d \boldx}$ &$=$ &$\boldW^T \frac{d \Loss}{d \boldy}$ & & $\frac{d \Loss}{d \boldX}$ &$=$ &$\boldW^T \frac{d \Loss}{d \boldY}$ \\
		$\frac{d \Loss}{d \boldW}$ &$=$ &$\frac{d \Loss}{d \boldy} \boldx^T$ & & $\frac{d \Loss}{d \boldW}$ &$=$ &$\frac{d \Loss}{d \boldY} \boldX^T$ \\
		$\frac{d \Loss}{d \boldb}$ &$=$ &$\frac{d \Loss}{d \boldy}$          & & $\frac{d \Loss}{d \boldb}$ &$=$ &$\Sum_{i=1}^{\NumOfSamples}\frac{d \Loss}{d \boldY_i}$
	\end{tabular}
\end{table}

\begin{gather*}
\boldy = f(\boldV f(\boldV f( \dots (\boldV f(\boldV \boldh_0 + \boldW \boldx_0 + \boldb) + \boldW \boldx_1 + \boldb) \dots) +  \boldW \boldx_{k-2} + \boldb ) + \boldW \boldx_{k-1} + \boldb)
\end{gather*}
Требуется найти частные производные $\partial \Loss / \partial \boldW$, $\partial \Loss / \partial \boldV$, $\partial \Loss / \partial \boldb$ при заданной производной $\partial \Loss / \partial \boldy$.

Посмотрим на некоторый уровень рекуррентной сети. Этот уровень осуществляет преобразование вида
\begin{equation*}
\boldy = f(\boldV \boldh + \boldW \boldx + \boldb)
\end{equation*}
Далее за этим уровнем следует еще какое-то число точно таких же уровней. Далее допустим, что 
\begin{itemize}
	\item Нам известна производная функции потерь $\partial \Loss / \partial \boldy$ по выходу конкретно данного уровня
	\item Нам известны производные функции потерь $\Loss$ по $\boldW$, $\boldV$, $\boldb$ как параметрам последующих уровней.
\end{itemize}
Иными словами, функция потерь $\Loss$ представляется как функция от выхода данного слоя $\boldy$, и параметров $\boldW$, $\boldV$, $\boldb$:
$$
\Loss = \Loss(\boldy, \boldV, \boldW, \boldb),
$$
причем известны частные производные $\partial \Loss / \partial \boldy$, $\partial \Loss/ \partial \boldW$, $\partial \Loss / \partial \boldV$, $\partial \Loss / \partial \boldb$. Тогда полные производные по $\boldW$, $\boldV$, $\boldb$, которые учитывают зависимость $\boldy$ от $\boldW$, $\boldV$, $\boldb$ имеют вид:
\begin{gather*}
\boldy = \boldy(\boldx, \boldh, \boldW, \boldV, \boldb)
\end{gather*}


\subsubsection{Покомпонентные производные}
\begin{align}
\frac{d\Loss}{d\boldW_{i,j}} & = \frac{d\Loss(\boldy, \boldV, \boldW, \boldb)}{d\boldW_{i,j}} = \partder{\Loss}{\boldW_{i, j}} + \left(\partder{\boldy}{\boldW_{i,j}}\right)^T \partder{\Loss}{\boldy} =  \partder{\Loss}{\boldW_{i, j}} + \boldx_j f'(\boldz_i) \partder{\Loss}{\boldy_i} \\ 
\frac{d\Loss}{d\boldV_{i,j}} & = \frac{d\Loss(\boldy, \boldV, \boldW, \boldb)}{d\boldV_{i,j}} = \partder{\Loss}{\boldV_{i, j}} + \left(\partder{\boldy}{\boldV_{i,j}}\right)^T \partder{\Loss}{\boldy} = \partder{\Loss}{\boldV_{i, j}} + \boldh_j f'(\boldz_i) \partder{\Loss}{\boldy_i} \\\
\frac{d\Loss}{d\boldb_{i}} & = \frac{d\Loss(\boldy, \boldV, \boldW, \boldb)}{d\boldb_{i}} = \partder{\Loss}{\boldb_i} + \left(\partder{\boldy}{\boldb_i}\right)^T \partder{\Loss}{\boldy} = \partder{\Loss}{\boldb_i} + f'(\boldz_i)\partder{\Loss}{\boldy_i} \ \\
\frac{d\Loss}{d\boldx_{i}} & = \frac{d\Loss(\boldy, \boldV, \boldW, \boldb)}{d\boldx_{i}} = \left(\partder{\boldy}{\boldx_i}\right)^T\partder{\Loss}{\boldy} =
(\boldW^i)^T \diag (f'(\boldz)) \partder{\Loss}{\boldy} \\
\frac{d\Loss}{d\boldh_{i}} & = \frac{d\Loss(\boldy, \boldV, \boldW, \boldb)}{d\boldh_{i}} = \left(\partder{\boldy}{\boldh_i}\right)^T\partder{\Loss}{\boldy} =
(\boldV^i)^T \diag (f'(\boldz))  \partder{\Loss}{\boldy} \\
\end{align}
Таким образом, требуется найти частные производные $\partder{\boldy}{\boldW_{i,j}}$, $\partder{\boldy}{\boldV_{i,j}}$, $\partder{\boldy}{\boldb_{i}}$
\begin{align*}
\partder{\boldy}{\boldW_{i,j}} &= \left(\frac{d\boldy}{d\boldz}\right)^T \frac{\partial\boldz}{\partial\boldW_{i, j}} =  \boldx_j \diag(f'(\boldz))\bolde_i =  \boldx_j  f'(\boldz_i) \bolde_i. \\
\frac{\partial \boldy}{\partial \boldV_{i, j}} &= \left(\frac{d\boldy}{d\boldz}\right)^T \frac{\partial\boldz}{\partial\boldV_{i, j}}  = \boldh_j \diag(f'(\boldz))  \bolde_i= \boldh_j f'(\boldz_i) \bolde_i.\\ 
\frac{\partial \boldy}{\partial \boldx_{i}} &= \left(\frac{d\boldy}{d\boldz}\right)^T \frac{\partial\boldz}{\partial\boldx_{i}} = \diag(f'(\boldz)) \boldW^i\\
\frac{\partial \boldy}{\partial \boldh_{i}} &= \left(\frac{d\boldy}{d\boldz}\right)^T \frac{\partial\boldz}{\partial\boldh_{i}} = \diag(f'(\boldz)) \boldV^i\\
\frac{\partial \boldy}{\partial \boldb_{i}} &= f'(\boldz_i)\bolde_i   \\
\end{align*}

%\begin{gather*}
%\left(}\right)^T \partder{\Loss}{\boldy} = 
%\end{gather*}

\subsubsection{Производные по матрицам и векторам}
\begin{align*}
\frac{d\Loss}{d\boldW} & = \partder{\Loss}{\boldW} + \diag(f'(\boldz))  \partder{\Loss}{\boldy} \boldx^T \\ 
\frac{d\Loss}{d\boldV} & = \partder{\Loss}{\boldV} + \diag(f'(\boldz))  \partder{\Loss}{\boldy} \boldh^T \\
\frac{d\Loss}{d\boldb} & = \partder{\Loss}{\boldb} + \diag(f'(\boldz))  \partder{\Loss}{\boldy} \\ 
\frac{d\Loss}{d\boldx} & = \boldW^T \diag (f'(\boldz)) \partder{\Loss}{\boldy}  \\
\frac{d\Loss}{d\boldh} & = \boldV^T \diag (f'(\boldz)) \partder{\Loss}{\boldy} 
\end{align*}

\subsubsection{Minibatch-производные}
\begin{align*}
\frac{d\Loss}{d\boldW} & = \partder{\Loss}{\boldW} + f'(\boldZ) \odot \partder{\Loss}{\boldy} \boldX^T \\ 
\frac{d\Loss}{d\boldV} & = \partder{\Loss}{\boldV} + f'(\boldZ) \odot \partder{\Loss}{\boldy} \boldH^T \\
\frac{d\Loss}{d\boldb} & = \partder{\Loss}{\boldb} + \sum_{i=1}^{\NumOfSamples} f'(\boldZ^i) \odot \partder{\Loss}{\boldY^i} \\ 
\frac{d\Loss}{d\boldX} & = \boldW^T \left(f'(\boldZ) \odot \partder{\Loss}{\boldY}\right)  \\
\frac{d\Loss}{d\boldH} & = \boldV^T \left(f'(\boldZ) \odot \partder{\Loss}{\boldY}\right) 
\end{align*}

\chapter{TensorFlow}

\begin{itemize}
	\item Device placement - изучить достоинства размещения переменных вручную
\end{itemize}

\bibliographystyle{ugost2008}
\bibliography{../bib/phy.bib}

\end{document}